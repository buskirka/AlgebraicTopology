\documentclass{article}

\title{Algebraic Topology Homework 5}
\author{Adam Buskirk}

\usepackage{amssymb,amsmath,amsthm}
\usepackage{tikz}
\usepackage[margin=1in]{geometry}

\newtheorem{theorem}[subsection]{Theorem}
\newtheorem{conjecture}[subsection]{Conjecture}
\newtheorem{lemma}[subsection]{Lemma}
\theoremstyle{definition}
\newtheorem{definition}[subsection]{Definition}

\newcommand{\R}{\mathbb{R}}
\newcommand{\N}{\mathbb{N}}
\newcommand{\Q}{\mathbb{Q}}
\newcommand{\Z}{\mathbb{Z}}
\newcommand{\Co}{\mathbb{C}}
\newcommand{\p}[1]{\left(#1\right)}
\newcommand{\sq}[1]{\left[#1\right]}
\newcommand{\set}[1]{\left\{#1\right\}}
\newcommand{\norm}[1]{\left|\left|#1\right|\right|}
% \newcommand{\p}[1]{\left(#1\right)}

\begin{document}
\maketitle

\section{$n$-manifolds}
\begin{lemma}\label{lemma1}
Every manifold has a countable basis of coordinate balls.
\end{lemma}
\begin{proof}
Suppose $X$ is a manifold, and consider an arbitrary open
$U \subseteq X$. Since $X$ is a manifold, it is second countable
and thus has a countable 
basis $\mathcal{B}$, and for some $B \in \mathcal{B}$,
$B \subseteq U$. For any $p \in B \subseteq U$ (we assume we pick this
consistently for each $B$; i.e. we have a choice function on $\mathcal{B}$) 
we also have a 
coordinate chart $(C_B,\varphi_B)$, $p \in C_B$, $C_B$ homeomorphic by
$\varphi_B$ to an open neighborhood of $\R^n$. Since $\varphi_B$ is 
a homeomorphism and $B$ is open, $\varphi(C_B \cap B)$ is open in
$\R^n$, and further $\varphi_B(p) \in \varphi_B(C_B \cap B)$. Since 
$\varphi_B(p)$ is an element in an open set, it must permit some open
ball around it, 
$V_B=B(\varphi_B(p); \epsilon) \subseteq \varphi_B(C_B \cap B)$. Then
$\varphi_B^{-1}(V_B) \subseteq C_B \cap B$, and 
$(\varphi_B^{-1}(V_B),\varphi_B|_{\varphi^{-1}(V_B)})$ is a coordinate 
ball chart, and at least one of these fits within our arbitrary $U$.
Consequently the collection
of all such coordinate balls, 
\[ \set{ (\varphi^{-1}(V_B),\varphi_B|_{\varphi^{-1}(V_B)}) : B \in \mathcal{B}}\]
is a countable basis of coordinate balls.
\end{proof}

\begin{theorem}
An $n$-manifold has a countable basis of precompact coordinate balls.
\end{theorem}
\begin{proof}
Suppose $X$ is a manifold. Then by Lemma \ref{lemma1}, 
$X$ permits a countable basis of coordinate balls.
\end{proof}

\section{Compactness implies paracompactness}
\begin{theorem}
Compactness implies paracompactness.
\end{theorem}
\begin{proof}
Suppose a space $X$ is compact. Then for any cover $\mathcal{B}
=\set{B_i, i\in I}$ over $X$,
then by compactness we have a finite subcover 
$\mathcal{B}'=\set{B_i, I\in J} \subseteq \mathcal{B}$. Since each 
$B_i \in \mathcal{B}'$ is an open subset of $B_i \in \mathcal{B}$, 
$\mathcal{B}'$ is an open refinement of $\mathcal{B}$. Furthermore, since
$\mathcal{B}'$ is finite, it must be locally finite. Thus
every open cover of $X$ admits a locally finite open refinement, so $X$
is paracompact.
\end{proof}

\end{document}
