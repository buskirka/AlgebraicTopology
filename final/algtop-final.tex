\documentclass{article}

\title{Algebraic Topology Final}
\author{Adam Buskirk}

\usepackage{amssymb,amsmath,amsthm}
\usepackage{tikz}
\usepackage{enumerate}
\usepackage[margin=1.5in]{geometry}

\newtheorem{theorem}{Theorem}
\newtheorem{problem}[theorem]{Problem}
\newtheorem{exercise}[theorem]{Exercise}
\newtheorem{conjecture}[theorem]{Conjecture}
\newtheorem{proposition}[theorem]{Proposition}
\newtheorem{lemma}[theorem]{Lemma}
\theoremstyle{definition}
\newtheorem{definition}[theorem]{Definition}

\newcommand{\R}{\mathbb{R}}
\newcommand{\N}{\mathbb{N}}
\newcommand{\Q}{\mathbb{Q}}
\newcommand{\Z}{\mathbb{Z}}
\newcommand{\Co}{\mathbb{C}}
\newcommand{\p}[1]{\left(#1\right)}
\newcommand{\sq}[1]{\left[#1\right]}
\newcommand{\set}[1]{\left\{#1\right\}}
\newcommand{\abs}[1]{\left|#1\right|}
\newcommand{\norm}[1]{\left|\left|#1\right|\right|}
% \newcommand{\p}[1]{\left(#1\right)}

\begin{document}
\null\hfill\parbox{40mm}{Algebraic Topology Final\par Adam Buskirk\par\today}

\section*{Theorem 5.20}
Almost entirely copied from the book, with slight reformatting and attempts at improving 
the clarity of the proof.
\renewcommand{\thetheorem}{5.20}
\begin{theorem}
Suppose $X_0 \subseteq X_1 \subseteq \cdots \subseteq X_{n-1} \subseteq X_n \subseteq \cdots$
is a sequence of topological spaces satisfying the conditions
\begin{enumerate}
\item $X_0$ is a nonempty discrete space
\item For each $n \ge 1$, $X_n$ is obtained from $X_{n-1}$ by attaching
a (possibly empty) collection of $n$-cells.
\end{enumerate}
Then $X=\bigcup_n X_n$ has a unique topology coherent with the family $\{X_n\}$, and a unique
cell decomposition making it into a CW complex whose $n$-skeleton is $X_n$ for each $n$.
\end{theorem}
\begin{proof}
\textbf{(Defining $X$'s topology)}
We give $X$ a topology by the criteria that $B \subseteq X$ is closed if and only if
$B \cap X_n$ is closed for each $n$. ``It is immediate that this is a topology, and it is
obviously the unique topology coherent with $\{X_n\}$''; if $B$ is closed in $X$, then
by definition $B \cap X_n$ is closed in $X_n$, and vice versa.%
\footnote{The text makes the argument
that if $B \subseteq X_n$ is closed, then Proposition 3.77 implies $X_{n-1}$ is a closed
subspace of $X_n$. Thus $B \cap X_{n-1}$ is also closed in $X_{n-1}$. 
Similarly, $B \cap X_{n+1} = B \cap X_n$ is closed in $X_{n+1}$. Thus $B\cap X_n$ would
be closed in each $X_n$ and thus $B$ would be closed in $X$. But this seems to assume that
any closed subset of $B$ must be entirely contained in some $X_n$, which does not seem to
follow. In particular, if $B$ contains exactly one point in $X_n \setminus X_{n-1}$ for 
each $n$ (when this is nonempty, e.g. $S^\infty$), 
then it would be closed in the topology given yet would not be a subset of 
any $X_n$.}

\textbf{(Obtaining $X$'s cell decomposition)}
To obtain a cell decomposition for $X$, we first let the $0$-cells be the points in
$X_0$. Then for each $n \ge 1$, we define a quotient map realizing $X_n$ as an adjunction
space 
\[
q_n : X_{n-1} \amalg \p{ \coprod_{\alpha \in A_n} D_\alpha^n} \to X_n
\]
Then by Proposition 3.77, we know $X_n \setminus X_{n-1}$ is an open subset of $X_n$
homeomorphic to $\coprod_{\alpha} \operatorname{Int} D_\alpha^n$ which is a disjoint
union of open $n$-cells, so we can define the $n$-cells of $X$ to be the components
$\{ e_\alpha^n \}$
of $X_n \setminus X_{n-1}$. $X$ can thus be expressed as the disjoint union of all of
its cells.

For each $n$-cell $e_\beta^n$ we define a characteristic map by composition to be
\[
\Phi_\beta^n 
: D_\beta^n 
\hookrightarrow X_{n-1} \amalg \p{ \coprod_{\alpha\in A_n} D_\alpha^n }
\overset{q_n}{\to} X_n
\hookrightarrow X
\]
``Clearly $\Phi_\beta^n$ maps $\partial D_\beta^n$ into $X_{n-1}$, and its restriction to 
$\operatorname{Int} D_\beta^n$ is a bijective continuous map onto $e_\beta^n$,
so we need only show that this restriction is a homeomorphism onto its image.
This follows because $\Phi_\beta^n|_{\operatorname{Int} D_\beta^n}$ is equal to
the inclusion of $\operatorname{Int} D_\beta^n$ into the disjoint union, followed by the 
restriction of $q_n$ to the saturated open subset $\operatorname{Int} D_\beta^n$,
which is a bijective quotient map onto $e_\beta^n$. This proves that $X$ has a cell 
decomposition for which $X_n$ is the $n$-skeleton for each $n$. Because the $n$-cells
of any such decomposition are the components of $X_n \setminus X_{n-1}$, this is the unique
such cell decomposition.''

\textbf{($X$ is Hausdorff)}
To show that $X$ is Hausdorff, we show that for arbitrary $p \in X$, 
there exists a continuous function $f : X \to [0,1]$
such that $f^{-1}(\{0\}) = \{p\}$ (using Exercise 2.35). Let $e_{\alpha_0}^m$ be the unique
cell containing $p$ of dimension $m$. We start by defining $f_m : X_m \to [0,1]$
as follows: if $m=0$, let $f_m(p)=0$ and $f_m(x)=1$ for $x\neq p$. Otherwise, let
$\tilde{p} = (\Phi_{\alpha_0}^m)^{-1}(p) \in \operatorname{Int} D_{\alpha_0}^m$ be
the preimage of $p$ in its closed cell $D_{\alpha_0}^m$. On $D_{\alpha_0}^m$, we
may define a continuous function $F : D_{\alpha_0}^m \to [0,1]$ which is equal to 
$1$ on $\partial D_{\alpha_0}^m$ and $0$ at exactly $\tilde{p}$ (using the result of
Problem 5-2(a)). We then extend $F$ to a function
\[
\tilde{f}_m : X_{m-1} \amalg \p{\coprod_{\alpha} D_\alpha^m} \to \R
\]
by letting $\tilde{f}_m$ be $F$ on $D_{\alpha_0}^m$ and $1$ everywhere else. $\tilde{f}_m$
is ``continuous by the characteristic property of the disjoint union, and descends to the
quotient to yield a continuous function $f_m : X_m \to [0,1]$ whose zero set is $\{p\}$.''

We now extend $f_m$ beyond $X_m$ using induction. If we have a continuous function
$f_{n-1} : X_{n-1} \to [0,1]$ with zero set exactly $\{p\}$, then we construct
$\tilde{f}_n$ to agree with $f_{n-1}$. We observe that we can define a function 
$F_\alpha^n : D_\alpha^n \to [0,1]$ which extends 
$f_{n-1} \circ \Phi_\alpha^n|_{\partial D_\alpha^n} : \partial D_\alpha^n \to [0,1]$
with no zeroes in $\operatorname{Int} D_\alpha^n$ (by Problem 5-2(b)).
We then define 
\[
\tilde{f}_n : X_{n-1} \amalg \p{\coprod_\alpha D_\alpha^n} \to \R
\]
by letting $\tilde{f}_n = f_{n-1}$ on $X_{n-1}$ (as we would expect from an extension)
and $\tilde{f}_n = F^n_\alpha$ on $D_\alpha^n$. Again, $\tilde{f}_n$ is continuous and
descends through the quotient to yield $f_n : X_n \to [0,1]$ whose zero set is exactly
$\{p\}$.

By letting $f(x) = f_n(x)$ iff $x \in X_n$, we obtain a continuous function 
(by Proposition 5.2) from $X$ to $[0,1]$ with zero set exactly $\{p\}$. This
is well defined by construction, since $f_n|_{X_{n-1}} = f_{n-1}$. Thus,
$X$ is a Hausdorff space, and since it can be equipped with the cell decomposition
described above it is a cell complex.

\textbf{(Each $X_n$ satisfies (C))}
By induction, in parallel with (W). 
It is obvious that $X_0$ satisfies (C) and (W), since it is a discrete space.
If we assume $X_{n-1}$ satisfies (C) and (W), note that for any $k$, 
$1 \le k \le n$, $\Phi_\alpha^k (\partial D_\alpha^k)$ is a compact subset of the 
CW complex $X_{k-1}$ and therefore by Theorem 5.14 it is contained in a finite 
subcomplex of $X_{k-1}$. Therefore $X_n$ satisfies (C).


\textbf{(Each $X_n$ satisfies (W))}
By induction, in parallel with (C).
It is obvious that $X_0$ satisfies (C) and (W), since it is a discrete space. 
Suppose $X_{n-1}$ satisfies (C) and (W). Then suppose $B \subseteq X_n$ has a closed
intersection with $\bar{e}$ for each cell $e$ in $X_n$. Then 
$B \cap X_{n-1}$ is closed in $X_{n-1}$ because $X_{n-1}$ satisfies (W), and 
$B \cap \overline{e_\alpha^n}$ is closed in $\overline{e_\alpha^n}$ for every such
cell by assumption, so $q_n^{-1}(B)$ is closed in 
$X_{n-1} \amalg (\coprod_\alpha D_\alpha^n)$, so $B$ is closed in $X_n$ by definition
of the quotient topology. Thus $X_n$ satisfies (W).

\textbf{($X$ satisfies (C))}
This follows directly from the fact that each $X_n$ satisfies (C), since each
cell of $X$, and the closure of that cell, are entirely contained in $X_n$.
$X$ satisfies (C).

\textbf{($X$ satisfies (W))}
Suppose $B \cap \bar{e}$ is closed for every cell $e$ in $X$. Then for each $n$,
$B \cap X_n$ is closed in $X_n$ since $X_n$ satisfies (W) and is composed of a
subset of $X$'s cells. Since $B \cap X_n$ is closed for each $n$, by the definition
of the topology on $X$ we know $B$ is closed in $X$. $X$ satisfies (W).

\textbf{($X$ is a CW complex)}
Since $X$ is a cell complex which satisfies conditions (C) and (W), $X$ is 
a CW complex.
\end{proof}

\section*{Supporting results}
\renewcommand{\thetheorem}{2.35}
\begin{exercise}
Suppose $X$ is a topological space, and for every $p \in X$ there exists a continuous
function $f : X \to \R$ such that $f^{-1}(0) = \{p\}$. Then $X$ is Hausdorff.
\end{exercise}

\renewcommand{\thetheorem}{3.77}
\begin{proposition}
Let $X \cup_f Y$ be an adjunction space, and let $q : X \sqcup Y \to X \cup_f Y$ be the
associated quotient map.
\begin{enumerate}[(a)]
\item The restriction of $q$ to $X$ is a topological embedding, whose image set $q(X)$ is 
a closed subspace of $X \cup_f Y$.
\item The restriction of $q$ to $Y \setminus A$ is a topological embedding, whose image set
$q(Y \setminus A)$ is an open subspace of $X \cup_f Y$.
\item $X \cup_f Y$ is the disjoint union of $q(X)$ and $q(Y \setminus A)$.
\end{enumerate}
\end{proposition}
(The text does not specify what $A$ actually is. I assume it is the points which get 
``adjuncted'', since that would make the most sense considering (b) and (c).)

\renewcommand{\thetheorem}{5.2}
\begin{proposition}
Suppose $X$ is a topological space whose topology is coherent with a family
$\mathcal{B}$ of subspaces.
\begin{enumerate}[(a)]
\item If $Y$ is another topological space, then a map $f : X \to Y$ is continuous 
if and only if $f|_B$ is continuous for every $B \in \mathcal{B}$.
\item The map $\coprod_{B \in \mathcal{B}} B \to X$ induced by inclusion of each 
set $B \hookrightarrow X$ is a quotient map.
\end{enumerate}
\end{proposition}

\renewcommand{\thetheorem}{5-2}
\begin{problem}
Suppose $D$ is a closed $n$-cell, $n \ge 1$.
\begin{enumerate}[(a)]
\item Given any point $p \in \operatorname{Int} D$, there is a continuous function
$F : D \to [0,1]$ such that $F^{-1}(1) = \partial D$ and $F^{-1}(0)=\{p\}$.
\item Given a continuous function $f : \partial D \to [0,1]$, $f$ extends to a
continuous function $F : D \to [0,1]$ that is strictly positive in 
$\operatorname{Int} D$.
\end{enumerate}
\end{problem}
\end{document}
