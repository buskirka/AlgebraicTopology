\documentclass{article}

\title{Algebraic Topology Exercises 3}
\author{Adam Buskirk}

\usepackage{amssymb,amsmath,amsthm}
\usepackage{tikz}
\usepackage[margin=1in]{geometry}

\newtheorem{theorem}[subsection]{Theorem}
\newtheorem{conjecture}[subsection]{Conjecture}
\newtheorem{lemma}[subsection]{Lemma}
\theoremstyle{definition}
\newtheorem{definition}[subsection]{Definition}

\newcommand{\R}{\mathbb{R}}
\newcommand{\N}{\mathbb{N}}
\newcommand{\Q}{\mathbb{Q}}
\newcommand{\Z}{\mathbb{Z}}
\newcommand{\Co}{\mathbb{C}}
\newcommand{\p}[1]{\left(#1\right)}
\newcommand{\sq}[1]{\left[#1\right]}
\newcommand{\set}[1]{\left\{#1\right\}}
\newcommand{\norm}[1]{\left|\left|#1\right|\right|}
% \newcommand{\p}[1]{\left(#1\right)}

\begin{document}
\maketitle

\begin{quote}
Show $\bar{B}_n$ is a manifold with boundary.
\end{quote}
\begin{theorem}
$\bar{B}_n$ is a manifold with boundary.
\end{theorem}
\begin{proof}
We can define a chart on $\bar{B}_n$ by defining, for each 
$(s,k) \in \{ -1, 1 \} \times \{ 1 \cdots n \}$, the map 
\[ 
\psi_{s,k} : (x_1 , \cdots , x_n) 
\mapsto s \cdot (x_1, \cdots, x_{k-1}, x_{k+1}, \cdots, x_n, x_k) 
\] 
This
is a linear transformation of $\R^n$, and further it is an isometry,
a symmetry in $\R^n$.
In particular, for each 
$C_{s,k} = \{ x \in \bar{B}_n : s \cdot x_k > 0\}$, we have
$\psi_{s,k} (C_{s,k}) = C_{1,n}$, and $C_{s,k} \cong S_{1,n}$. 
In turn, $C_{1,n} \cong \{ x : 0 < x_n \le 1\}$, since we have the 
homeomorphism 
\[
\phi : (x_1, \cdots, x_n) 
\mapsto 
\frac{ |(x_1,\cdots,x_n)|}{x_n} \cdot (x_1, \cdots, x_n)
\]
which is composed of continuous functions, and thus is continuous. Furthermore,
this is homeomorphic to the half-space $H^n=\{ x : x_n \ge 0 \}$ by the map 
$\xi : (x_1,\cdots,x_n) \mapsto (x_1, \cdots, -\ln x_n)$. 
Thus, $C_{s,k} \cong C_{1,n} \cong \{ x : 0 < x_n \le 1 \} \cong H^n$, and our
atlas is
\[ 
\{(B_1(0), I_R)\} 
\cup \set{ (C_{s,k},\xi \circ \phi \circ \psi_{s,k}) 
: s \in \{1,-1\}\times\{1,\cdots,n\} } 
\]
Since every point in $\bar{B}_n$ is in one of these charts into $\R^n$ or $H^n$---%
each interior point is
in $B_1(0)$ and everything but $0$ is in at least one $C_{s,k}$---this is, indeed,
an atlas for $\bar{B}_n$ and thus it is a manifold with boundary.
\end{proof}

\begin{quote}
Is $S^1 \times \bar{B}_2$ a manifold with boundary?
\end{quote}
It would seem so. If $x \in S^1 \times \bar{B}_2$, then $x=(x_1, x_2)$
for some $x_1 \in S^1$ and $x_2 \in \bar{B}_2$, around which there must exist 
open sets $U_1\subseteq S^1$ and $U_2\subseteq\bar{B}_2$ which are homeomorphic 
subsets of either $\R$ ($\R^2$ for $\bar{B}_2$) 
or $[0,\infty)$ ($H^2$ for $\bar{B}_2$). Then by Proposition 3.33 the product of these 
homeomorphism is a homeomorphism. Since $S^1$ is a manifold, $U_1$ must be 
homeomorphic to a subset of $\R$, and thus the product homeomorphism must be 
either a homeomorphism into a subset of $\R \times \R^2 \cong \R^3$ or 
$\R \times H^2 \cong H^3$.
\end{document}
