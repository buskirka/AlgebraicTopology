\documentclass{article}

\title{Algebraic Topology HW6}
\author{Adam Buskirk}

\usepackage{amssymb,amsmath,amsthm}
\usepackage{tikz}
\usepackage[margin=1in]{geometry}

\newtheorem{theorem}[subsection]{Theorem}
\newtheorem{conjecture}[subsection]{Conjecture}
\newtheorem{lemma}[subsection]{Lemma}
\theoremstyle{definition}
\newtheorem{definition}[subsection]{Definition}

\newcommand{\R}{\mathbb{R}}
\newcommand{\N}{\mathbb{N}}
\newcommand{\Q}{\mathbb{Q}}
\newcommand{\Z}{\mathbb{Z}}
\newcommand{\Co}{\mathbb{C}}
\newcommand{\p}[1]{\left(#1\right)}
\newcommand{\sq}[1]{\left[#1\right]}
\newcommand{\set}[1]{\left\{#1\right\}}
\newcommand{\abs}[1]{\left|#1\right|}
\newcommand{\norm}[1]{\left|\left|#1\right|\right|}
% \newcommand{\p}[1]{\left(#1\right)}

\begin{document}
\maketitle

Problems 4.18, 4.19, 4.33.

``Smooth Bump Function'':

Given $A \subseteq U \subseteq \R$, $A$ closed, $U$ open.

(S) Show $\exists f : \R \to \R$ such that 
\begin{enumerate}
\item $f(A)=\{1\}$
\item $\operatorname{supp} f \subseteq U$
\item $f \in C^\infty (\R)$
\end{enumerate}

(A) Replace (3) with $f$ analytic on $\R$. Can we construct 
such a bump function?

\section{Problem 4.18}
Let $M_1$ and $M_2$ be $n$-manifolds. For $i=1,2$, let 
$B_i \subseteq M_i$ be regular coordinate balls, and let
$M_i' = M_i \setminus B_i$. Choose a homeomorphism 
$f : \partial M_2' \to \partial M_1'$. Let 
$M_1 \# M_2$ be the adjunction space of $M_1' \cup_f M_2'$.
\subsection{Part 4.18.a}
\begin{theorem}
$M_1 \# M_2$ is an $n$-manifold (without boundary).
\end{theorem}
\begin{proof}
This follows quite directly from Theorem 3.79 in the text, since $M_1 \# M_2$ is an
adjunction space of two manifolds with boundary, $M_1'$ and $M_2'$ with
a homeomorphism between their boundaries.
\end{proof}
\subsection{Part 4.18.b}
\begin{theorem}
If $M_1$ and $M_2$ are connected and $n>1$, then $M_1 \# M_2$ is connected.
\end{theorem}
\begin{proof} (Using path connectedness)
First we show that $M_1'$ and $M_2'$ are path-connected. Without loss of generality,
we consider $M_1$. Then $M_1'=M_1 \setminus B_1$ is a manifold with boundary 
$\partial M_1' \cong S^{n-1}$. Since $M_1$ is a connected manifold, it is path connected,
so for any two points $x,y \in M_1' \subseteq M_1$ there is a path $f : [0,1] \to M_1$ 
in $M_1$ between them. Furthermore, $S^{n-1}$ is path-connected since $n>1$.
Then since $f^{-1}(\partial M_1')$ is a preimage of a closed set, it contains its
infinum $a$ and its supremum $b$. Then since $B_1 \cong S^{n-1}$ is path 
connected, there exists a path $g : [0,1] \to \partial M_1'$ connecting these two.
These two paths can be mashed into a single new path connecting $x$ and $y$,
\[
h(t) = \left\{ \begin{array}{cl}
f(t) & t \in [0,1] \setminus (a,b) \\
g\p{\frac{t-a}{b-a}} & t \in (a,b) 
\end{array} \right.
\]
This essentially takes the path $f$, but then takes a detour around \textit{all}
of the set $B_1$ through the boundary of $M_1'$. Thus, any two points in $M_1'$
have a path $h(t)$ as above between them, and thus $M_1'$ (and $M_2'$ similarly)
are path-connected. Similarly, their images under the quotient are path-connected, 
and these share a point (under the boundary), and thus their union, which is
$M_1 \# M_2$, is path connected by Proposition 4.13.b. Since $M_1 \# M_2$ is
path connected, and all manifolds are locally path connected, by Proposition 4.26.e, 
$M_1 \# M_2$ is connected.
\end{proof}
\subsection{Part 4.18.c}
\begin{theorem}
If $M_1$ and $M_2$ are compact, then $M_1 \# M_2$ is compact.
\end{theorem}
\begin{proof}
$M_1'$ and $M_2'$ are the complements of open sets of $M_1$ and $M_2$, and thus
are closed, and since $M_1$ and $M_2$ are compact, $M_1'$ and $M_2'$ are compact
by Proposition 4.36.a. Thus, $M_1' \sqcup M_2'$ is compact; any open cover $\beta$ 
of $M_1' \sqcup M_2'$ must have finite subcovers of $M_1'$ and $M_2'$, denoted
$\beta_1$ and $\beta_2$ respectively, and then $\beta_1 \cup \beta_2$ is a finite
subcover of $M_1' \sqcup M_2'$. $M_1 \# M_2$ is a quotient space of this disjoint
union, and thus by Proposition 4.36.e, $M_1 \# M_2$ is compact.
\end{proof}

\section{Problem 4.19}
\begin{theorem}
Let $M_1 \# M_2$ be a connected sum of $n$-manifolds $M_1$ and $M_2$. Then
there are open subsets $U_1, U_2 \subseteq M_1 \# M_2$ and points $p_i \in 
M_i$ such that 
\begin{enumerate}
\item $U_i \cong M_i \setminus \{p_i\}$
\item $U_1 \cap U_2 \cong \R^n \setminus \{0\}$, and
\item $U_1 \cup U_2 = M_1 \# M_2$.
\end{enumerate}
\end{theorem}
\begin{proof}
Recall that $M_1$ and $M_2$ will be joined along a \textit{regular}
coordinate ball. Thus, in $M_1$ we have an open neighborhood $B_1'$ homeomorphic 
to $\R^n$ containing this coordinate ball $B_1$ which is mapped to $B^n$ by this 
homeomorphism $f_1$. Similarly, in $M_2$ we have $B_1' \cong_{f_2} \R^n$ such 
that $f_2(B_2) = B^n$. By deleting $f_2^{-1}(\{0\})$ from $B_2$, we can invert the 
space along its boundary using the automorphism on $\R^n \setminus \{0\}$ defined
by $x \mapsto x/\norm{x}^2$. 
\end{proof}
\section{Problem 4.33}
\section{Smooth Bump Function (S)}
\section{Smooth Bump Function (A)}

\end{document}
