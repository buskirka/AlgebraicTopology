\documentclass{article}

\title{Algebraic Topology Homework 2}
\author{Adam Buskirk}

\usepackage{amssymb,amsmath,amsthm}
\usepackage{tikz}
\usepackage[margin=1in]{geometry}
\usepackage{enumerate}

\newtheorem{theorem}[subsection]{Theorem}
\newtheorem{conjecture}[subsection]{Conjecture}
\newtheorem{lemma}[subsection]{Lemma}
\theoremstyle{definition}
\newtheorem{definition}[subsection]{Definition}

\newcommand{\R}{\mathbb{R}}
\newcommand{\N}{\mathbb{N}}
\newcommand{\Q}{\mathbb{Q}}
\newcommand{\Z}{\mathbb{Z}}
\newcommand{\Co}{\mathbb{C}}
\newcommand{\p}[1]{\left(#1\right)}
\newcommand{\sq}[1]{\left[#1\right]}
\newcommand{\set}[1]{\left\{#1\right\}}
\newcommand{\norm}[1]{\left|\left|#1\right|\right|}
% \newcommand{\p}[1]{\left(#1\right)}

\begin{document}
\maketitle

\section{Local Euclidean definitions}
\begin{quote}
Show we can replace $V$ in the definition of local Euclidean with
$B_1(0)$ or $R^n$.
\end{quote}
\begin{theorem}
The following three definitions are equivalent:
\begin{enumerate}[(A)]
\item $M$ is \textbf{locally Euclidean of dimension $n$} if every point of $M$ has a 
neighborhood of $M$ that is homeomorphic to an open subset of $\R^n$.
\item $M$ is \textbf{locally Euclidean of dimension $n$} if every point of $M$ has a 
neighborhood of $M$ that is homeomorphic to $B_1(0) \subseteq \R^n$.
\item $M$ is \textbf{locally Euclidean of dimension $n$} if every point of $M$ has a 
neighborhood of $M$ that is homeomorphic to $\R^n$.
\end{enumerate}
\end{theorem}
\begin{proof}
(A to B) Suppose $M$ satisfies definition (A), and consider an arbitrary point 
$x \in M$. Then there exists $U$ such that $x \in U$ and 
$\varphi : U \to V \subseteq \R^n$ is a homeomorphism. Since $V$ is open in $\R^n$,
there exists an open ball $B_r(\varphi(x))$ in $V$ around the image of $x$. Since open
sets in $\R^n$ are closed under affine linear transformations, applying the map
$\phi : z \mapsto r^{-1} I_n (z-\varphi(x))$, which acts on the open ball $B_r(\varphi(x))$
by first translating it to $B_r(0)$ and then scaling it by a factor of $r^{-1}$ to
$B_1(0)$; hence $\phi$ is a homeomorphism from $B_r(\varphi(x))$ to $B_1(0)$. Then we know
that there is a neighborhood $U'=\varphi^{-1}(B_r(\varphi(x))) \subseteq M$ which is 
homeomorphic to $B_1(0)$ by the composition of homeomorphisms 
$\phi \circ \sq{\varphi|_{U'}}$. Hence $M$ also satisfies definition (B).

(B to C) Example 2.25 in the text demonstrates that $B_1(0) \subseteq \R^n$ is 
homeomorphic to $\R^n$; let this map be called $\phi$. Then if $M$ satisfies
definition (B), around any $x \in M$ there exists $U$ containing $x$, with
$\varphi : U \to B_1(0)$. Composing these two maps gives us $\phi \circ \varphi$,
a homeomorphism from that same $U$ around $x$ into $\R^n$. Thus (B) implies (C).

(C to A) $\R^n$ is an open subset of $\R^n$, and thus the same $U$ and homeomorphism 
$\varphi : U \to \R^n$ guaranteed by (C) may serve for the definition of (A) as well.
\end{proof}

\end{document}
