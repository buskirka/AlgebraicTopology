\documentclass{article}

\title{Algebraic Topology Homework 1}
\author{Adam Buskirk}

\usepackage{amssymb,amsmath,amsthm}
\usepackage{tikz}
\usepackage[margin=1in]{geometry}

\newtheorem{theorem}[subsection]{Theorem}
\newtheorem{conjecture}[subsection]{Conjecture}
\newtheorem{lemma}[subsection]{Lemma}
\theoremstyle{definition}
\newtheorem{definition}[subsection]{Definition}

\newcommand{\R}{\mathbb{R}}
\newcommand{\N}{\mathbb{N}}
\newcommand{\Q}{\mathbb{Q}}
\newcommand{\Z}{\mathbb{Z}}
\newcommand{\p}[1]{\left(#1\right)}
\newcommand{\set}[1]{\left\{#1\right\}}
% \newcommand{\p}[1]{\left(#1\right)}

\begin{document}
\maketitle
\section{Problem 1}
\begin{definition}[Topological basis, def.\ A]\label{topbasisA}
A \textbf{basis} for a topological space $(X,\tau)$ is a collection $\beta \subseteq \tau$ 
such that for any $U \in \tau$ and $x \in U$, there exists some $B \in \beta$ such that
$x \in B \subseteq U$.
\end{definition}
\begin{definition}[Topological basis, def.\ B]\label{topbasisB}
A \textbf{basis} for a topological space $(X,\tau)$ is a collection $\beta \subseteq \tau$ such that
\begin{enumerate}
\item $\beta$ covers $X$; that is, $\bigcap_{B \in \beta} B = X$, and
\item any $U \in \tau$ can be written as a union of sets in $\beta$.
\end{enumerate}
\end{definition}
\begin{theorem}
Definitions \ref{topbasisA} and \ref{topbasisB} are equivalent.
\end{theorem}
\begin{proof}
For the sake of discussion we will refer to a collection $\beta \subseteq\tau$ 
matching Definition \ref{topbasisA} as a \textbf{A-basis} 
and one matching Definition \ref{topbasisB} as a \textbf{B-basis}.

(A$\to$B)

Suppose that $\beta\subset\tau$ is an A-basis for $(X,\tau)$. Then for any $x \in X$, there must
exist some $B \in \tau$ such that $x \in B \subseteq X$. Hence every element of $X$ is in some
element of $\beta$; $\beta$ covers $X$, the first condition of \ref{topbasisB}. 

Consider then an arbitrary $U \in \tau$; for each $x \in U$, let $B_x$ denote the set guaranteed 
by \ref{topbasisA} such that $x \in B_x \subseteq U$. Then observe that 
$\bigcup_{x\in U} B_x = U$; if $x_0 \in U$, then $x_0 \in B_{x_0} \subseteq \bigcup_{x\in U} B_x$
and thus $U \subseteq \bigcup_{x\in U} B_x$. If $x_0 \in U^C$, then $x_0 \not\in B_x$ for each
$x\in U$. Consequently, $x_0 \not\in\bigcup_{x\in U} B_x$. Hence $U^C \subseteq \p{\bigcup_{x\in U} B_x}^C$.

(B$\to$A)

\end{proof}


\end{document}
