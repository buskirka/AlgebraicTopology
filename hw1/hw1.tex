\documentclass{article}

\title{Algebraic Topology Homework 1}
\author{Adam Buskirk}

\usepackage{amssymb,amsmath,amsthm}
\usepackage{tikz}
\usepackage[margin=1in]{geometry}
\usepackage{enumerate}

\newtheorem{theorem}[subsection]{Theorem}
\newtheorem{conjecture}[subsection]{Conjecture}
\newtheorem{lemma}[subsection]{Lemma}
\theoremstyle{definition}
\newtheorem{definition}[subsection]{Definition}

\newcommand{\R}{\mathbb{R}}
\newcommand{\N}{\mathbb{N}}
\newcommand{\Q}{\mathbb{Q}}
\newcommand{\Z}{\mathbb{Z}}
\newcommand{\p}[1]{\left(#1\right)}
\newcommand{\set}[1]{\left\{#1\right\}}
% \newcommand{\p}[1]{\left(#1\right)}

\begin{document}
\maketitle
\section{Exercise 2.9}
\begin{theorem}
Let $X$ be a topological space and let $A \subseteq X$ be any subset.
\begin{enumerate}[(a)]
\item A point is in $\operatorname{Int} A$ iff it has a neighborhood contained in $A$.
\item A point is in $\operatorname{Ext} A$ iff it has a neighborhood contained in $A^C$.
\item A point is in $\partial A$ iff every neighborhood contains a point of $A$ and a point of $A^C$.
\item A point is in $\bar{A}$ iff every neighborhood of it contains a point of $A$. 
\item $\bar{A} = A \cup \partial A = \operatorname{Int} A \cup \partial A$.
\item $\operatorname{Int} A$ and $\operatorname{Ext} A$ are open in $X$, while $\bar{A}$
    and $\partial A$ are closed in $X$.
\item The following are equivalent:
    \begin{enumerate}[(i)]
    \item $A$ is open in $X$.
    \item $A=\operatorname{Int} A$
    \item $A$ contains none of its boundary points.
    \item Every point of $A$ has a neighborhood contained in $A$.
    \end{enumerate}
\item The following are equivalent:
    \begin{enumerate}[(i)]
    \item $A$ is closed in $X$.
    \item $A=\bar{A}$.
    \item $A$ contains all of its boundary points.
    \item Every point of $X\setminus A$ has a neighborhood 
        contained in $X\setminus A$.
    \end{enumerate}
\end{enumerate}
\end{theorem}
\begin{proof}
(a, $\Rightarrow$) 
If $x \in \operatorname{Int} A$, which is defined as a 
union of open subsets of $A$, then $x$ must have been contained in one of 
these open sets, $N_x$, which is a neighborhood of $x$ contained in $A$.

(a, $\Leftarrow$) 
Suppose $x$ has an open neighborhood $N_x$ contained in $A$. 
Then $N_x$ is an element of the union defining $\operatorname{Int} A$, and 
thus $ x \in N_x \subseteq \operatorname{Int} A$.

(b, $\Rightarrow$)
Suppose $x \in \operatorname{Ext} A$. Then $x \in X \setminus \bar{A}$. Thus
there must exist some closed set $B \supseteq A$ such that $x \not\in B$.
So $x \in B^C \subseteq A^C$, with $B^C$ the complement of a closed set, and thus
open.

(b, $\Leftarrow$)
Suppose $x \in X$ has an open neighborhood $N_x$ such that $N_x \subseteq A^C$. 
Since $N_x$ is open, $(N_x)^C$ is closed, and contains $A$. Thus it is part
of the intersection defining $\bar{A}$, so
$\bar{A} \subseteq (N_x)^C$. Thus $\bar{A}^C = X \setminus \bar{A} 
=\operatorname{Ext} A \supseteq N_x$, which contains $x$. Thus
$x \in \operatorname{Ext} A$.

(c, $\Rightarrow$)
Suppose $x \in \partial A$. Then $x$ is in neither the interior of $A$ nor
its exterior. Then consider an arbitrary neighborhood of $x$, $N_x$. If 
$N_x$ did not contain an element of $A^C$, $N_x$ would be a subset
of the interior of $A$ (it would be part of the union in the interior) 
and thus $x$ would be in the interior; contradiction, $N_x$ contains
an element of $A^C$. Then suppose $N_x$ contains no element of $A$. Then 
$N_x \subseteq A^C$; then $(A^C)^C = A \subseteq (N_x)^C$. So $(N_x)^C$
is part of the intersection in the closure; that is, 
$\bar{A} \subseteq (N_x)^C$. Thus $N_x \subseteq \bar{A}^C =
\operatorname{Ext} A$. But this implies $x$ is in the exterior of $A$;
contradiction. Thus $N_x$ contains both an element of $A$ and an element of
$A^C$.

(c, $\Leftarrow$)
Consider any $x \in X$ with the property that every neighborhood of it contains
an element of $A$ and an element of $A^C$. Then $x$ cannot be in the interior,
since then it would be in some open set $N_x$ in the interior's union, which
would also contain a point of $A^C$, which contradicts the requirement that
$N_X \subseteq A$. It also cannot be in the exterior if it were in the exterior,
then it would not be in the closure of $A$, which means there must exist some
$B \subseteq X$ such that $x\not\in B$, $B \supseteq A$ and $B$ is closed.
But then $x \in B^C$, which is open, and $B^C \subseteq A^C$. This $B$ cannot
contain any element of $A$, and yet this is a neighborhood of $x$, which 
contradicts our assumptions regarding $x$. Thus, if every neighborhood of
$x$ contains elements of both $A$ and $A^C$, then $x$ must be a boundary point.

(d, $\Rightarrow$)
Suppose $x \in \bar{A}$. Either $x \in A$ or not. If $x \in A$, then every
neighborhood of it must contain a point of $A$. If $x \not\in A$, then
$x \in \partial A$ or $x \in \operatorname{Ext} A$. The latter would imply
the existence of an open neighborhood $N_x$ of $x$ contained entirely in $A^C$,
implying that (closed) $(N_x)^C$ entirely contained $A$, and hence it is 
part of the intersection defining the closure; so $x \not\in (N_x)^C$ which
contains $\bar{A}$, hence $x \not\in \bar{A}$. Contradiction. If, instead,
$x \in \partial A$, (c) guarantees the point required.

(d, $\Leftarrow$)
Suppose every neighborhood of some $x \in X$ contains a point of $A$. Then
if $x \not\in \bar{A}$, we would have some closed set around $A$ which
did not contain $x$. The complement of this would be an open neighborhood
containing $x$, which must thus contain a point of $A$, contradicting
the fact that it cannot include any point of $A$. Hence $x \in\bar{A}$.

(e)
Suppose $x \in \bar{A}$. Then either $x \in A$ or $x \not\in A$ but every
neighborhood contains an element of $A$, which means it contains both elements
of $A$ and $A^C$, hence by (c) $x \in \partial A$. Thus 
$\bar{A} \subseteq A \cup \partial A$.

Suppose $x \in A \cup \partial A$. If $x \in A$ then either every $N_x$ 
containing $x$ contains an element of $A^C$, and thus $x \in \partial A$ (c),
or not, which means there exists an $N_x$ containing no point of $A^C$, meaning
$x \in \operatorname{Int} A$ (a). Thus 
$A \cup \partial A \subseteq \operatorname{Int} A \cup \partial A$.

Suppose $x \in \operatorname{Int} A \cup \partial A$. 
$\operatorname{Int} A \subseteq A \subseteq \bar A$, and if $x \in \partial A$
then each neighborhood contains a point of $A$ (c) which means it is also 
in $\bar{A}$ (d). Hence $\operatorname{Int} A \cup \partial A \subseteq \bar{A}$.

These three inclusions can be strung together to demonstrate equality.

(f)
The interior is a union of open sets and thus open.
The exterior is a complement of a closed set $\bar{A}$ and thus is open.
The closure is an intersection of closed sets and thus is closed.
The boundary is the complement of the union of two open sets, which is open,
and thus the boundary is closed.

(g, i to ii)
Suppose $A$ is open in $X$. Then $A$ is one of the sets in the interior union and thus
$A = \operatorname{Int} A$.

(g, ii to iii)
Suppose $A = \operatorname{Int} A$. Then since $\partial A$ and $\operatorname{Int} A$
are disjoint, $A$ contains none of its boundary points.

(g, iii to iv)
Suppose $A$ contains none of its boundary points. Then if $x \in A$, not all neighborhoods
of $x$ contain a point of $A^C$. Any of these neighborhoods satisfies the fourth point.

(g, iv to i)
Suppose every point of $A$ has a neighborhood contained in $A$. Then the union of these
neighborhoods is a subset of $A$---indeed, equal to $A$, since each point was in at least
one of these neighborhoods---and this is a union of open sets and thus it is open, so 
$A$ is open.

(h, i to ii)
Suppose $A$ is closed in $X$. Then $A$ is one of the sets in the closure intersection,
and thus $A \subseteq \bar{A} \subseteq A$. $A=\bar{A}$.

(h, ii to iii)
Suppose $A = \bar{A}$. Then by (e) $A = \bar{A} = A \cup \partial A$ and thus $A$
contains all its boundary points.

(h, iii to iv) 
If every neighborhood around $x \in A^C$ was not contained in $A^C$, then 
$x$ would be in $\partial A$ by (c) which would contradict (h.iii). Hence
if (h.iii) is true then every point in $A^C$ must have at least one neighborhood
contained in $A^C$.

(h, iv to i)
Suppose every point $x \in A^C$ has a neighborhood $x \in N_x \subseteq A^C$. The 
complements of these sets are all closed sets containing all of $A$ and not $x$; 
intersecting all of these sets gives us a closed set which excludes every point
not in $A$ in at least one set of the intersection, but includes every point of
$A$. This set is thus $A$, and thus $A$ is a closed set (the intersection of 
a collection of closed sets).
\end{proof}

\section{Exercise 2.11}
Handled by Preheim.
%\begin{theorem}
%$A$ is dense in $X$ iff every nonempty open subset of $X$ contains a point of $A$.
%\end{theorem}
%\begin{proof}
%($\Rightarrow$) 
%Suppose $A$ is dense in $X$, and consider an arbitrary nonempty open subset $N$ of $X$. Then 
%$N^C$ is a closed set. If $N$ does not contain a point of $A$, then $N \subseteq A^C$, so
%$A \subseteq N^C$, and thus $N^C$ is a part of the intersection defining the closure of $A$. 
%Since $N$ contains some point, $N^C$ must lack some point of $X$, so 
%\end{proof}

\section{Exercise 2.32}
\begin{theorem}
\begin{enumerate}[(a)]
\item Every homeomorphism is a local homeomorphism.
\item Every local homeomorphism is continuous and open.
\item Every bijective local homeomorphism is a homeomorphism.
\end{enumerate}
\end{theorem}
\begin{proof}
(a) Suppose $f : X \to Y$ is a homeomorphism. Then $f$ is a local homeomorphism; 
for each point the trivial restriction of $f$ to $X$ itself is $f$, which is a 
homeomorphism.

(b) Suppose $f : X \to Y$ is a local homeomorphism. 

(b, open)
If $A \in \tau_X$ then for each $x \in A$, there exists $U_x$ such that $f|_{U_x}$
is a homeomorphism. $V_x = f|_{U_x}(A)$ is open, since open sets in open subspaces
are open in the original space (they are the intersection of two open sets and 
hence open) So $V_x = f|_{U_x}(A) = f(A \cap U_x) \subseteq f(A)$, and thus 
$\bigcup_{x \in A} V_x \subseteq f(A)$. Yet also $f(x) \in V_x$ for each $x$, so
$f(A) \subseteq \bigcup_{x\in A} V_x$. So $f(A)$ can be expressed as a union of
open sets, $A$ is open, and thus $f$ is an open map.

(b, continuous)
Consider $B \in \tau_Y$. Then either $f(X) \cap B = \emptyset$ or not; if so, 
$f^{-1}(B)=\emptyset$ which is open. Otherwise, for each $x \in f^{-1}(B)$, we
know $[f|_{U_x}]^{-1}(B)$ is open in $U_x$. Since $U_x$ is an open subspace, 
$K_x = [f|_{U_x}]^{-1} (B)$ is open in $X$ also. Then $K_x \subseteq f^{-1}(B)$ and
each $y \in B$ has at least one $x \in f^{-1}(B)$ mapping to it, which must be in
$K_x$. Hence $\bigcup_{x \in f^{-1}(B)} K_x = f^{-1} (B)$, and thus $f^{-1}(B)$
can be expressed as a union of open sets, $f^{-1}(B)$ is open, and $f$ is a 
continuous map.

%If $f$ is not open, then for some open set $A$ it maps $x \in A$ to a boundary point 
%of $f(A)$. Then, letting $f_x : U_x \to f(U_x)$ denote the restricted homeomorphism guaranteed by 
%local homeomorphism, we have that $A \cap U_x$ is open, and thus $f_x(A \cap U_x)$ is an open
%neighborhood of $f(x)$. But this is a subset of $f(A)$ and thus 

%If $A$ is an open set, then we know for each $x \in A$, there exists open $U_x$
%such that $f(U_x)$ is open in $Y$. Furthermore, $f|_{U_x}$ is a homeomorphism onto
%$f(U_x)$ with the subspace topology. $A \cap U_x$ is open, so 
%$f|_{U_x}(A \cap U_x)$ is open in $f(U_x)$. So there must be some open set $B_x$
%in $Y$ 
%such that $B_x \cap U_x = f|_{U_x}(A \cap U_x)$.

%If $f$ is not an open map, 
%then there exists some open set $A$ such that $f(A)$ is not open. So some point in 
%$A$, call it $x$, is mapped to a boundary point of $f(A)$. Then the local 
%homeomorphism guarantees the existence of $U_x$ around $x$ such that 
%$f|_{U_x}$ is a homeomorphism.

%then there exists some $B \subseteq Y$ such that 
%$f^{-1}( \operatorname{Int} B ) \nsupseteq \operatorname{Int} f^{-1}(B)$.
%That is, there is a point $x$ in 
%$\operatorname{Int} f^{-1} (B)$ which is not in 
%$f^{-1}( \operatorname{Int} B)$. Since $x$ is in the interior of the set, there 
%exists a neighborhood $N_x$ such that $x \in N_x \subseteq f^{-1}(B)$.
%The fact that $f$ is a local homeomorphism guarantees the existence of some 
%$V_x$ such that $f|_{V_x}$ is a homeomorphism. 

(c) Suppose $f : X \to Y$ is a bijective local homeomorphism.
Then by the result of part (b) above and the product of Exercise 2.29 in the
text, $f$ is a bijective continuous map which is open and thus a homeomorphism.
\end{proof}
\section{Exercise 2.40*}
\begin{definition}[Topological basis, def.\ A]\label{topbasisA}
A \textbf{basis} for a topological space $(X,\tau)$ is a collection $\beta \subseteq \tau$ 
such that for any $U \in \tau$ and $x \in U$, there exists some $B \in \beta$ such that
$x \in B \subseteq U$.
\end{definition}
\begin{definition}[Topological basis, def.\ B]\label{topbasisB}
A \textbf{basis} for a topological space $(X,\tau)$ is a collection $\beta \subseteq \tau$ such that
\begin{enumerate}
\item $\beta$ covers $X$; that is, $\bigcup_{B \in \beta} B = X$, and
\item any $U \in \tau$ can be written as a union of sets in $\beta$.
\end{enumerate}
\end{definition}
\begin{theorem}
Definitions \ref{topbasisA} and \ref{topbasisB} are equivalent.
\end{theorem}
\begin{proof}
For the sake of discussion we will refer to a collection $\beta \subseteq\tau$ 
matching Definition \ref{topbasisA} as a \textbf{A-basis} 
and one matching Definition \ref{topbasisB} as a \textbf{B-basis}.

(A$\to$B)

Suppose that $\beta\subseteq\tau$ is an A-basis for $(X,\tau)$. Then for any $x \in X$, there must
exist some $B \in \tau$ such that $x \in B \subseteq X$. Hence every element of $X$ is in some
element of $\beta$; $\beta$ covers $X$, the first condition of \ref{topbasisB}. 

Consider then an arbitrary $U \in \tau$; for each $x \in U$, let $B_x$ denote the set guaranteed 
by \ref{topbasisA} such that $x \in B_x \subseteq U$. Then observe that 
$\bigcup_{x\in U} B_x = U$; if $x_0 \in U$, then $x_0 \in B_{x_0} \subseteq \bigcup_{x\in U} B_x$
and thus $U \subseteq \bigcup_{x\in U} B_x$. If $x_0 \in U^C$, then $x_0 \not\in B_x$ for each
$x\in U$. Consequently, $x_0 \not\in\bigcup_{x\in U} B_x$. Hence 
$U^C \subseteq \p{\bigcup_{x\in U} B_x}^C$, and thus $\bigcup_{x\in U} B_x \subseteq U$. Thus,
$U = \bigcup_{x\in U}$. Hence any arbitrary open set in $\tau$ can be expressed as a union of sets
in $\beta$, completing the definition of a B-basis. All A-bases are also B-bases.

(B$\to$A)

Suppose that $\beta\subseteq\tau$ is a B-basis for $(X,\tau)$. Then suppose we are given some 
$U \in \tau$ and $x \in U$. Then by the definition of a B-basis, there exists some 
$\Gamma \subseteq \beta$ such that $\bigcup_{\gamma \in \Gamma} \gamma = U$. Since
$x = U = \bigcup_{\gamma\in\Gamma} \gamma$, $x \in \gamma_x$ for some 
$\gamma_x \in \Gamma \subseteq \beta$. 
Furthermore, by the nature of a union, $\gamma_x \subseteq U$. 
Hence for each $U$, and $x \in U$, there exists some 
$\gamma_x \in \beta$ such that $x \in \gamma_x \subseteq U$. Consequently, $\beta$ is also an
A-basis. All B-bases are also A-bases.

Definitions \ref{topbasisA} and \ref{topbasisB} imply each other, and thus they are equivalent.
\end{proof}

\section{Problem 2.1d}
\section{Problem 2.4}
\section{Problem 2.14}
\section{Problem 2.15}
\section{Problem 2.20}

\end{document}
